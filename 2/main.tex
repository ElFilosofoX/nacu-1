\documentclass{scrartcl}
\usepackage[utf8]{inputenc}
\usepackage{pgf-pie}
\usepackage{subfigure}
\usepackage{hyperref}
\usepackage{cite}
\usepackage{amsmath}
\usepackage{amssymb}
\usepackage{graphicx}
%\documentclass{book}
\renewcommand\thesubsection{\Alph{subsection}}


%opening
\title{Assignment 1}
\author{Daan Spijkers, s1011382\\ Tomás Catalán López, s1081589\\ Willem Lambooy, s1009584}

\begin{document}
\maketitle

\textbf{Note: you can find the GitHub repository at}
\url{github.com/dspkio/nacu}.

\subsection*{4}

\begin{itemize}

  \item[(a)]
    No, it is not. Our space of valid solutions is $\{\{e_1, e_2\}, \{e_2,
    e_3\}, \{e_3, e_4\}\}$. We see that $e_1$ and $e_4$ occur only once,
    and $e_2$, $e_3$ occur twice.

  \item[(b)]
    Not always. Our goal is finding the optimal solution as efficiently as
    possible, if a bias is helpful for that then it is not harmful. If,
    for example, in the example given in (a) the weights for $e_2$ and
    $e_3$ were \emph{less} than the weights for $e_1$ and $e_4$, then this
    would be a positive thing.

\end{itemize}

\subsection*{5}
The intuitive understanding is that ants going up will get a path straight
to the destination, while ants going down will get stuck, and have longer
paths. Even though the shortest path (length $5$) is down, it is possible
that the ants will converge on the upper path (length $8$).

We can make a gross estimation of the expected length of the upper and
lower paths. Up is clearly 8, but lower is more difficult to calculate.
However, we can calculate what the probability is of a path shorter or
equal to 8, as well as larger than 8, and provide a lower bound.

Given we initially go down, we have an approximately $\frac{1}{3} \times
(\frac{1}{2})^2 = \frac{1}{12}$ chance of the shortest path, of length 5.
Length 6 is also $\frac{1}{12}$, but length 7 is $\frac{1}{12} +
\frac{1}{24}$. Similarly, length 8 is $\frac{1}{24}$, and $>8$ is then $1
- \frac{4}{12} = \frac{2}{3}$. So, the expected length is bounded below:
$E(\text{down}) \ge \frac{2}{3} \times 9 + 2.125 = 8.125$. Which is larger
than 8, so it is quite possible for things to converge on the top path.
This is still random, obviously, and so it will depend on the randomness
which path it will actually find.

\end{document}
