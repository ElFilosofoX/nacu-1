\documentclass{scrartcl}
\usepackage[utf8]{inputenc}
\usepackage{amsmath}
\usepackage{amssymb}

%opening
\title{Assignment 1}
\author{Daan Spijkers, s1011382\\ Tomás Catalán López, s1081589\\ Willem Lambooy, s1009584}

\begin{document}

\maketitle

\section{}
Schemata A1 has more wildcards than A2: $o(A1) < o(A2)$. Since both schemata have the same amount of bits, the chance of a non-wildcard bit flipping in A1 is less than the chance of that happening in A2.

Mathematically, this can be computed as follows (with $p_m=0.01$):
\begin{itemize}
  \item Probability of a bit not being flipped after mutation: $1-p_m=0.99$.
  \item Probability that A1 survives: $S_m(A1)=(1-p_m)^{o(A1)}=0.99^4\approx0.961$.
  \item Probability that A2 survives: $S_m(A2)=(1-p_m)^{o(A2)}=0.99^5\approx0.951$.
\end{itemize}

\section*{7}
% the question says function, terminal set AND s-expression, but
% s-expression is just the function and terminal set, no? I'm not sure if
% I'm leaving something out, but I feel like this is fine -- daan
\begin{itemize}
  \item[(a)]
    $T = \{x, y, z, \text{true}, \text{false}\}$. False isn't
    \emph{strictly} necessary, but otherwise it would be a trivial
    formula.

    $F = \{\land, \lor, \leftrightarrow\}$. These all have arity 2.

  \item[(b)]
    $T = \mathbb{R} \cup \{x, z\}$. Technically $0.234$ and $0.789$ are in
    $\mathbb{Q}$, so that would suffice, but taking the reals is more
    standard.

    $F = \{+, -, *\}$. These all have arity 2.

\end{itemize}

\section*{8}

\end{document}
