\documentclass{scrartcl}
\usepackage[utf8]{inputenc}

%opening
\title{Assignment 1}
\author{Daan Spijkers, s1011382\\ Tomás Catalán López, s1081589\\ Willem Lambooy, s1009584}

\begin{document}

\maketitle

\begin{abstract}

\end{abstract}

\section{}
Schemata A1 has more wildcards than A2: $o(A1) < o(A2)$. Since both schemata have the same amount of bits, the chance of a non-wildcard bit flipping in A1 is less than the chance of that happening in A2.

Mathematically, this can be computed as follows (with $p_m=0.01$):
\begin{itemize}
 \item Probability of a bit not being flipped after mutation: $1-p_m=0.99$.
 \item Probability that A1 survives: $S_m(A1)=(1-p_m)^{o(A1)}=0.99^4\approx0.961$.
 \item Probability that A2 survives: $S_m(A2)=(1-p_m)^{o(A2)}=0.99^5\approx0.951$.
\end{itemize}


\end{document}
